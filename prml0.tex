%%%%%%%%%%%%%%%%%%%%%%%%%%%%%%%%%%%%%%%%%%%%%%%%%%%%%%%%%%%%%%%%%%%%%%%%%%%
% 第0章
%%%%%%%%%%%%%%%%%%%%%%%%%%%%%%%%%%%%%%%%%%%%%%%%%%%%%%%%%%%%%%%%%%%%%%%%%%%
\chapter*{まえがき}


サイボウズ・ラボでは「言語処理に必要そうな機械学習の基礎知識を身につける」という目標のもと,2011年の2月から11月にかけて当時シュプリンガー・ジャパン(現在は丸善)から出版されていた『パターン認識と機械学習』(以下PRML)を輪読する社内読書会をやっていました.
「あの本」を10ヶ月足らずで一通り(すべての章ではありませんが)読みきったと言えば,そのスパルタな様子が想像つくのではないでしょうか.しかも,専門の学生ではない社会人が仕事の合間に!

当然スムーズに読み進めるはずもなく,いろんなところでつまずくことになりました.一つには機械学習の考え方に慣れてなかったという部分がもちろんあげられますが,まさにそれを身につけるために読んでいるわけですから,そこはしかたありません.
そしてそれ以上に参加メンバーが四苦八苦したのは,やはり「計算」でした.
たとえばPRMLの2章ではベクトルや行列による偏微分という大技がいきなり炸裂しますし,行列の固有値は常識中の常識,積分の変数変換なんかはもう当然知ってるもの.そして全編に渡ってLagrangeの未定乗数法という謎の魔法に支配されている…….
数学科出身の約2名から「そこはなんでそんな計算になるの? 明らかにおかしいよね?」と突っ込まれながらPRMLの要所要所が省略された数式を追いかけるという経験は,世の中は「数学ガール」のようには行かないということをきっと教えてくれたことでしょう.

さて,
そんなメンバーの助けにと,同じく読書会に参加していた同僚の光成さん(@herumiさん)が,PRMLのアンチョコ(教科書ガイド)を作ってくれました.
それらは\url{https://herumi.github.io/prml/}にてCC-BY 3.0(クリエイティブ・コモンズ 表示 3.0 非移植)のライセンスで公開されています\footnote{このライセンスは原著作者のクレジットを表示すれば自由に複製・配布できます.また,著作物を二次加工したり,二次加工物を商用利用もできます.\url{http://creativecommons.org/licenses/by/3.0/deed.ja}.}.

このアンチョコは,PRML前半のキーポイントである2章から5章まで,そして後半の難関である9章と10章をカバーしています.出てくる数式を手抜き無しのガチンコで展開しつつ,
それらを理解するのに必要な数学の道具(積分の変数変換,行列の各種操作)なども平行して解説するという作りになっています.
線形代数と解析をまともにやったのは大学の教養課程が最後という三十路の技術者(数学をもっとちゃんとやっておけば良かった!)にはもちろん,
現役の学生さんにとっても,このアンチョコはなかなか役に立つでしょう.

こんな親切に説明されたら自分で考えなくなってしまうんじゃあないかと逆に不安になる,という方は一回目はアンチョコを写経して,次は見ずに自力で計算してみる,というのがおすすめ.

このアンチョコで挫折しないPRMLライフを楽しんでくださいね!

……と,普通ならここで話は終わるはずだったのですが,サイボウズ・ラボには竹迫さんという,いつも超本気で冗談をする人がいまして\footnote{2015年に株式会社リクルートマーケティングパートナーズに転職.},
いつのまにかこのPRMLアンチョコがPRMLによく似た装丁の同人誌(本物のISBNコード付き!)になっていたんです.といっても,いきなり何千冊も刷るなんて冒険はさすがにできなくて,言語処理学会での宣伝用に見本誌を数部作っただけでした.
が,これが各方面で思いのほか評判を呼び,本格的にまとまった部数作ろうじゃあないかという話になり,あれよあれよと,この「パターン認識と機械学習の学習」がこんな立派な形で今読んでいらっしゃるみなさんのお手元に届くことになりました.

というわけで社内読書会の言い出しっぺとして拙い序文を書かせていただきました.
最後に,厳しい読書会について来てくれたサイボウズ・ラボの同僚と,大元のPRML読書会を主催してくださった{\tt naoya\_t}さんおよび参加者のみなさんに感謝を捧げます.

\begin{flushright}

社内PRML読書会 主宰 : 中谷 秀洋
\end{flushright}

\vspace{\baselineskip}
\section*{著者より}
本書はPRMLに登場する数式を理解するために必要な数学をまとめたものです.
いくつかの定理は証明せずに認めますが, 可能な限り自己完結を目指しました.
概ねPRMLに従ってますが, 違う方法をとっているところもあります.
間違い, 質問などございましたら, {\tt herumi@nifty.com}または{\tt Twitter:@herumi}までご連絡ください.

2017年の普及版では五代さんが隅々まで細かい校正・修正作業をしてくださいました. この場を借りてお礼を申し上げます.

なおまえがきにある通り, この本のPDF版をまるごと無償で\url{https://herumi.github.io/prml/}にて公開しています.
「数式は紙に書かれたものを鉛筆で追わないと頭に入らない」人(私だ)でなければそちらでもよいでしょう.
\begin{flushright}

著者 : 光成 滋生
\end{flushright}
